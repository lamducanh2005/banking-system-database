\newminted[MySQLCode]{sql}{
    frame=single, % Viền bao quanh mã
    framesep=2mm, % Khoảng cách giữa mã và viền
    fontsize=\scriptsize, % Kích thước chữ nhỏ hơn  footnotesize
    baselinestretch=1.1, % Dãn dòng vừa phải
    breaklines, % Tự động xuống dòng
    tabsize=3, % Kích thước tab
    autogobble % Loại bỏ khoảng trắng thừa
}

\begin{itemize}

    \item Procedure \texttt{GetAccountsByCustomerID} hiển thị thông tin tài khoản của một khách hàng, thông qua mã khách hàng:
    \begin{MySQLCode}
    CREATE PROCEDURE GetAccountsByCustomerID(
        IN p_customerID VARCHAR(50)
    )
    BEGIN
        SELECT accountNumber, accountName, balance, productCode, openingBranch, openingDate
        FROM accounts
        WHERE customerID = p_customerID;
    END;
    \end{MySQLCode}

    \item Procedure \texttt{GetTransactionHistory} hiển thị lịch sử giao dịch của một tài khoản, thông qua số tài khoản:
    \begin{MySQLCode}
    CREATE PROCEDURE GetTransactionHistory(
        IN p_accountNumber VARCHAR(50)
    )
    BEGIN
        SELECT transactionNumber, sourceAccount, targetAccount, transactionType, amount, time, description
        FROM transactions
        WHERE sourceAccount = p_accountNumber OR targetAccount = p_accountNumber
        ORDER BY time DESC;
    END;
    \end{MySQLCode}

    \item Procedure \texttt{GetEmployeesByBranch} lấy danh sách nhân viên của một chi nhánh, thông qua mã chi nhánh:
    \begin{MySQLCode}
    CREATE PROCEDURE GetEmployeesByBranch(
        IN p_branchCode VARCHAR(50)
    )
    BEGIN
        SELECT employeeCode, CONCAT(lastName, ' ', firstName) as FullName, position, department, managerCode
        FROM employees
        WHERE branchCode = p_branchCode;
    END;
    \end{MySQLCode}

    \item Procedure \texttt{GetTransactionsBySpecificDate} đưa ra danh sách các giao dịch trong một ngày, tháng hoặc năm cụ thể:
    \begin{MySQLCode}
    CREATE PROCEDURE GetTransactionsBySpecificDate(
        IN p_year INT,      -- Có thể NULL
        IN p_month INT,     -- Có thể NULL
        IN p_day INT        -- Có thể NULL
    )
    BEGIN
        SELECT transactionNumber, sourceAccount, targetAccount, transactionType, amount, time, description, employeeCode
        FROM transactions
        WHERE
            (p_year IS NULL OR YEAR(time) = p_year) AND
            (p_month IS NULL OR MONTH(time) = p_month) AND
            (p_day IS NULL OR DAY(time) = p_day)
        ORDER BY time;
    END;
    \end{MySQLCode}

    \item Procedure \texttt{AddNewCustomer} thực hiện thêm một khách hàng mới vào hệ thống:
    \begin{MySQLCode}
    CREATE PROCEDURE AddNewCustomer(
        IN p_customerID VARCHAR(50),
        IN p_lastName VARCHAR(50),
        IN p_firstName VARCHAR(50),
        IN p_birthDate DATE,
        IN p_phoneNumber VARCHAR(15),
        IN p_address VARCHAR(256),
        IN p_creditScore INT
    )
    BEGIN
        -- Kiểm tra nếu khách hàng đã tồn tại
        IF EXISTS (SELECT 1 FROM customers WHERE customerID = p_customerID) THEN
            SIGNAL SQLSTATE '45000'
                SET MESSAGE_TEXT = 'Khách hàng đã tồn tại trên hệ thống';
        ELSE
            INSERT INTO customers (customerID, lastName, firstName, birthDate, phoneNumber, address, creditScore)
            VALUES (p_customerID, p_lastName, p_firstName, p_birthDate, p_phoneNumber, p_address, p_creditScore);
        END IF;
    END;
    \end{MySQLCode}

    \item Procedure \texttt{OpenNewAccount} thực hiện mở một tài khoản mới trên hệ thống và thêm giao dịch nạp tiền khi mở tài khoản thành công:
    \begin{MySQLCode}
    CREATE PROCEDURE OpenNewAccount(
        IN p_accountNumber VARCHAR(256),
        IN p_accountName VARCHAR(256),
        IN p_balance BIGINT,
        IN p_customerID VARCHAR(50),
        IN p_productCode VARCHAR(50),
        IN p_openingBranch VARCHAR(50),
        IN p_openingDate DATETIME
    )
    BEGIN
        -- Thêm tài khoản mới
        INSERT INTO accounts (accountNumber, accountName, balance, customerID, productCode, openingBranch, openingDate)
        VALUES (p_accountNumber, p_accountName, p_balance, p_customerID, p_productCode, p_openingBranch, p_openingDate);

        -- Thêm giao dịch nạp tiền khi mở tài khoản
        INSERT INTO transactions (sourceAccount, targetAccount, transactionType, amount, time, description, employeeCode)
        VALUES (NULL, p_accountNumber, 'OPENING', p_balance, NOW(), 'Mo tai khoan', NULL);
    END;
    \end{MySQLCode}

    \item Procedure \texttt{GetEmployeesByPosition} lấy danh sách các nhân viên, thông qua một chức vụ nhất định:
    \begin{MySQLCode}
    CREATE PROCEDURE GetEmployeesByPosition(
        IN p_position VARCHAR(256)
    )
    BEGIN
        SELECT employeeCode, lastName, firstName, branchCode, department
        FROM employees
        WHERE position = p_position;
    END;
    \end{MySQLCode}

    \newpage
    
    \item Procedure \texttt{CalculateInterestForAccount} thực hiện tính lãi suất theo chu kỳ cho một tài khoản:
    \begin{MySQLCode}
    CREATE PROCEDURE CalculateInterestForAccount(
        IN p_accountNumber VARCHAR(256)
    )
    BEGIN
        DECLARE p_balance BIGINT;               -- Số dư tài khoản
        DECLARE p_interestRate FLOAT;           -- Lãi suất (\%)
        DECLARE p_term INT;                     -- Kỳ hạn (tháng)
        DECLARE last_interest_date DATETIME;    -- Ngày trả lãi gần nhất
        DECLARE interest_amount BIGINT;         -- Số tiền lãi
        DECLARE next_interest_date DATETIME;    -- Ngày trả lãi kế tiếp
        DECLARE product_type VARCHAR(50);       -- Loại tài khoản
        DECLARE count_withdrawal INT;           -- Biến đếm giao dịch rút tiền
        DECLARE amount_withdrawal BIGINT;       -- Số tiền đã rút hoặc chuyển đi

        -- Bước 1: Lấy thông tin tài khoản
        SELECT a.balance, p.interestRate, p.term, p.productType
        INTO p_balance, p_interestRate, p_term, product_type
        FROM accounts a
        JOIN products p ON a.productCode = p.productCode
        WHERE a.accountNumber = p_accountNumber;

        -- Bước 2: Xác định ngày trả lãi gần nhất
        SELECT MAX(t.time)
        INTO last_interest_date
        FROM transactions t
        WHERE (t.sourceAccount = p_accountNumber OR t.targetAccount = p_accountNumber)
            AND t.transactionType IN ('INTEREST', 'OPENING');

        -- Bước 3: Tính ngày trả lãi kế tiếp
        SET next_interest_date = DATE_ADD(last_interest_date, INTERVAL p_term MONTH);

        -- Bước 4: Tính lãi suất và kiểm tra giao dịch rút tiền
        WHILE next_interest_date <= CURDATE() DO
            -- Kiểm tra giao dịch rút tiền trong khoảng thời gian này
            SELECT COUNT(*)
            INTO count_withdrawal
            FROM transactions t
            WHERE t.transactionType IN ('WITHDRAW', 'TRANSFER')
                AND t.time BETWEEN last_interest_date AND next_interest_date
                AND (t.sourceAccount = p_accountNumber);

            -- Nếu có giao dịch rút tiền, không tính lãi cho kỳ hạn này
            IF count_withdrawal > 0 THEN
                -- Tính toán số tiền đã rút hoặc chuyển đi (sử dụng giao dịch rút tiền trong kỳ hạn)
                SELECT SUM(t.amount)
                INTO amount_withdrawal
                FROM transactions t
                WHERE t.transactionType IN ('WITHDRAW', 'TRANSFER')
                    AND t.time BETWEEN last_interest_date AND next_interest_date
                    AND t.sourceAccount = p_accountNumber;

                -- Cập nhật số dư sau mỗi lần rút tiền hoặc chuyển khoản
                SET p_balance = p_balance - amount_withdrawal;
    \end{MySQLCode}
    \newpage
    \begin{MySQLCode}
            ELSE
                -- Nếu không có giao dịch rút tiền, tính lãi cho kỳ hạn này
                -- Tính tiền lãi = Số dư * Lãi suất * Thời gian gửi (tháng) / 12
                SET interest_amount = ROUND(p_balance * p_interestRate * p_term / 12);

                -- Kiểm tra loại tài khoản
                IF product_type = 'Credit' OR product_type = 'Checking' THEN
                    -- Thu lãi
                    INSERT INTO transactions (sourceAccount, targetAccount, transactionType, amount, time, description, employeeCode)
                    VALUES (p_accountNumber, NULL, 'INTEREST', ABS(interest_amount), next_interest_date, 'Ngan hang thu lai', NULL);
                ELSE
                    -- Trả lãi
                    INSERT INTO transactions (sourceAccount, targetAccount, transactionType, amount, time, description, employeeCode)
                    VALUES (NULL, p_accountNumber, 'INTEREST', interest_amount, next_interest_date, 'Ngan hang tra lai', NULL);
                END IF;
                -- Cập nhật số dư tài khoản
                SET p_balance = p_balance + interest_amount;
            END IF;
            -- Cập nhật ngày trả lãi kế tiếp
            SET next_interest_date = DATE_ADD(next_interest_date, INTERVAL p_term MONTH);
        END WHILE;

        -- Bước 5: Cập nhật số dư tài khoản cuối cùng
        UPDATE accounts
        SET balance = p_balance
        WHERE accountNumber = p_accountNumber;
    END;
    \end{MySQLCode}
    
\end{itemize}