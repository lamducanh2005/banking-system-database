\newminted[MySQLCode]{sql}{
    frame=single, % Viền bao quanh mã
    framesep=2mm, % Khoảng cách giữa mã và viền
    fontsize=\scriptsize, % Kích thước chữ nhỏ hơn  footnotesize
    baselinestretch=1.1, % Dãn dòng vừa phải
    breaklines, % Tự động xuống dòng
    tabsize=3, % Kích thước tab
    autogobble % Loại bỏ khoảng trắng thừa
}

TRUY VẤN KHÁCH HÀNG:
\begin{itemize}

    \item Truy vấn xem mỗi khách hàng có bao nhiêu tài khoản (group by + inner join + aggregate functions)
    \begin{MySQLCode}
    SELECT 
        c.customerID,
        CONCAT(c.lastName, ' ', c.firstName) AS customerName,
        COUNT(a.accountNumber) AS totalAccounts
    FROM 
        bank_db.customers c
    JOIN 
        bank_db.accounts a ON c.customerID = a.customerID
    GROUP BY 
        c.customerID;
    \end{MySQLCode}

    \item Truy vấn tính tổng số tiền trong tất cả tài khoản của mỗi khách hàng (group by + inner join + aggregate functions)
    \begin{MySQLCode}
    SELECT 
        c.customerID,
	CONCAT(c.lastName, ' ', c.firstName) AS customerName,
        SUM(a.balance) AS totalBalance
    FROM 
        bank_db.customers c
    JOIN 
        bank_db.accounts a ON c.customerID = a.customerID
    GROUP BY 
	c.customerID;
    \end{MySQLCode}

    \item Truy vấn tìm khách hàng có sinh nhật vào quý 4
    \begin{MySQLCode}
    SELECT 
        customerID,
        CONCAT(lastName, ' ', firstName) AS customerName,
        birthDate
    FROM 
        bank_db.customers
    WHERE 
        MONTH(birthDate) IN (10, 11, 12);
    \end{MySQLCode}

    \item Truy vấn tìm khách hàng có sinh nhật vào quý 4
    \begin{MySQLCode}
    SELECT 
        customerID,
        CONCAT(lastName, ' ', firstName) AS customerName,
        birthDate
    FROM 
        bank_db.customers
    WHERE 
        MONTH(birthDate) IN (10, 11, 12);
    \end{MySQLCode}

     \item Truy vấn số lượng khách hàng ở mỗi thang điểm tín dụng (group by + aggregate functions)
    \begin{MySQLCode}
    SELECT 
        creditScore,
        COUNT(*) AS customerCount
    FROM 
        bank_db.customers
    GROUP BY 
        creditScore
    ORDER BY 
        creditScore;
    \end{MySQLCode}

    \item Truy vấn mỗi khách hàng đã thực hiện bao nhiêu giao dịch (inner join + group by + aggregate functions)
    \begin{MySQLCode}
    SELECT 
        c.customerID,
        CONCAT(c.lastName, ' ', c.firstName) AS customerName,
        COUNT(*) AS totalTransactions
    FROM
	bank_db.customers c
    JOIN
        bank_db.accounts a ON c.customerID = a.customerID
    JOIN 
        bank_db.transactions t ON a.accountNumber = t.sourceAccount OR a.accountNumber = t.targetAccount
    GROUP BY 
        c.customerID;
    \end{MySQLCode}

    \item Truy vẫn tất cả tài khoản và số dư trong đó của mỗi khách hàng (outer join + aggregate functions)
    \begin{MySQLCode}
    SELECT 
        c.customerID,
        CONCAT(c.lastName, ' ', c.firstName) AS customerName,
        a.accountNumber,
        a.balance
    FROM 
        bank_db.customers c
    LEFT JOIN 
        bank_db.accounts a ON c.customerID = a.customerID

    UNION

    SELECT 
        c.customerID,
        CONCAT(c.lastName, ' ', c.firstName) AS customerName,
        a.accountNumber,
        a.balance
    FROM 
        bank_db.customers c
    RIGHT JOIN 
        bank_db.accounts a ON c.customerID = a.customerID;
    \end{MySQLCode}

    \item Truy vấn mỗi khách hàng đã được phục vụ bởi bao nhiêu nhân viên (inner join + group by + aggregate functions)
    \begin{MySQLCode}
    SELECT 
        c.customerID,
        CONCAT(c.lastName, ' ', c.firstName) AS customerName,
        COUNT(DISTINCT t.employeeCode) AS totalEmployees
    FROM 
        bank_db.customers c
    JOIN 
        bank_db.accounts a ON c.customerID = a.customerID
    JOIN 
        bank_db.transactions t ON a.accountNumber = t.sourceAccount OR a.accountNumber = t.targetAccount
    GROUP BY 
        c.customerID;
    \end{MySQLCode}

\end{itemize}
TRUY VẤN TÀI KHOẢN:
\begin{itemize}

    \item Truy vấn mỗi tài khoản đã thực hiện bao nhiêu giao dịch chuyển tiền (group by + inner join + aggregate functions)
    \begin{MySQLCode}
    SELECT 
        a1.accountNumber,
        a1.accountName,
        COUNT(t1.transactionNumber) AS totalTransactions
    FROM 
        bank_db.accounts a1
    JOIN 
        bank_db.transactions t1 ON a1.accountNumber = t1.sourceAccount
    GROUP BY 
        a1.accountNumber;
    \end{MySQLCode}

    \item Truy vấn mỗi tài khoản đã thực hiện chuyển hay rút bao nhiêu tiền (group by + inner join + aggregate functions)
    \begin{MySQLCode}
    SELECT 
        a1.accountNumber,
        a1.accountName,
        SUM(t1.amount) AS totalTransactionsAmount
    FROM 
        bank_db.accounts a1
    JOIN 
        bank_db.transactions t1 ON a1.accountNumber = t1.sourceAccount
    GROUP BY 
        a1.accountNumber;
    \end{MySQLCode}

    \item Truy vấn mỗi tài khoản đã chuyển nhiều tiền nhất cho tài khoản nào khác null (subquery trong from và where + aggregate functions + group by)
    \begin{MySQLCode}
    SELECT 
        t1.sourceAccount, 
        t1.targetAccount, 
        t1.totalAmount
    FROM (
        SELECT 
            sourceAccount, 
            targetAccount, 
            SUM(amount) AS totalAmount
        FROM 
            bank_db.transactions
        WHERE 
            sourceAccount IS NOT NULL 
            AND targetAccount IS NOT NULL 
            AND amount IS NOT NULL
        GROUP BY 
            sourceAccount, targetAccount
    ) t1
    WHERE 
        t1.totalAmount = (
            SELECT 
                MAX(t2.totalAmount)
            FROM (
                SELECT 
                    sourceAccount, 
                    targetAccount, 
                    SUM(amount) AS totalAmount
                FROM 
                    bank_db.transactions
                WHERE 
                    sourceAccount IS NOT NULL 
                    AND targetAccount IS NOT NULL 
                    AND amount IS NOT NULL
                GROUP BY 
                    sourceAccount, targetAccount
            ) t2
            WHERE 
                t2.sourceAccount = t1.sourceAccount
        );
    \end{MySQLCode}

    \item Truy vẫn mỗi tài khoản đã thực hiện nhiều giao dịch chuyển tiền nhất cho tài khoản nào khác null (subquery trong from và where + aggregate functions + group by)
    \begin{MySQLCode}
    SELECT 
        t1.sourceAccount, 
        t1.targetAccount, 
        t1.transactionCount
    FROM (
        SELECT 
            sourceAccount, 
            targetAccount, 
            COUNT(*) AS transactionCount
        FROM 
            bank_db.transactions
        WHERE 
            sourceAccount IS NOT NULL 
            AND targetAccount IS NOT NULL
        GROUP BY 
            sourceAccount, targetAccount
    ) t1
    WHERE 
        t1.transactionCount = (
            SELECT 
                MAX(t2.transactionCount)
            FROM (
                SELECT 
                    sourceAccount, 
                    targetAccount, 
                    COUNT(*) AS transactionCount
                FROM 
                    bank_db.transactions
                WHERE 
                    sourceAccount IS NOT NULL 
                    AND targetAccount IS NOT NULL
                GROUP BY 
                    sourceAccount, targetAccount
            ) t2
            WHERE 
                t2.sourceAccount = t1.sourceAccount
        );
    \end{MySQLCode}

    \item Truy vấn số lượng tài khoản được mở mỗi năm (group by + aggregate functions)
    \begin{MySQLCode}
    SELECT 
        YEAR(openingDate) AS openingYear,
        COUNT(*) AS accountCount
    FROM 
        bank_db.accounts
    WHERE 
        openingDate IS NOT NULL
    GROUP BY 
        YEAR(openingDate)
    ORDER BY 
        openingYear;
    \end{MySQLCode}

\end{itemize}
TRUY VẤN NHÂN VIÊN:
\begin{itemize}

   \item Truy vấn nhân viên và tất cả người quản lý của họ (self join + aggregate functions)
    \begin{MySQLCode}
    SELECT 
        e.employeeCode,
        CONCAT(e.lastName, ' ', e.firstName) AS employeeName,
        m.employeeCode AS managerCode,
        CONCAT(m.lastName, ' ', m.firstName) AS managerName
    FROM 
        bank_db.employees e
    LEFT JOIN 
        bank_db.employees m ON e.managerCode = m.employeeCode;
    \end{MySQLCode}

    \item Truy vấn nhân viên và số cấp dưới họ quản lý (self join + group by + aggregate functions)
    \begin{MySQLCode}
    SELECT 
        e1.employeeCode AS managerCode,
        CONCAT(e1.lastName, ' ', e1.firstName) AS managerName,
        COUNT(e2.employeeCode) AS subordinateCount
    FROM 
        bank_db.employees e1
    LEFT JOIN 
        bank_db.employees e2 ON e1.employeeCode = e2.managerCode
    GROUP BY 
        e1.employeeCode
    ORDER BY 
        subordinateCount DESC;
    \end{MySQLCode}

    \item Truy vấn nhân viên và tổng số tiền giao dịch họ đã xử lý (inner join + group by + aggregate functions)
    \begin{MySQLCode}
    SELECT 
        e.employeeCode,
        CONCAT(e.lastName, ' ', e.firstName) AS employeeName,
	SUM(t.amount) AS totalRevenue
    FROM 
        bank_db.employees e
    JOIN 
        bank_db.transactions t ON t.employeeCode = e.employeeCode
    GROUP BY 
        e.employeeCode
    ORDER BY 
        totalRevenue DESC
    \end{MySQLCode}

    \item Truy vấn mỗi nhân viên đã từng phục vụ bao nhiêu khách hàng (inner join + group by + aggregate functions)
    \begin{MySQLCode}
    SELECT 
        e.employeeCode,
        CONCAT(e.lastName, ' ', e.firstName) AS employeeName,
        COUNT(*) AS totalCustomers
    FROM
	bank_db.employees e
    JOIN
        bank_db.transactions t ON e.employeeCode = t.employeeCode
    JOIN 
        bank_db.accounts a ON a.accountNumber = t.sourceAccount OR a.accountNumber = t.targetAccount
    GROUP BY 
        e.employeeCode;
    \end{MySQLCode}

\end{itemize}
TRUY VẤN GIAO DỊCH:
\begin{itemize}

    \item Truy vấn tính số lượng giao dịch trong mỗi năm (group by + aggregate functions)
    \begin{MySQLCode}
    SELECT 
        YEAR(time) AS Year,
        COUNT(*) AS transactionCount
    FROM 
        bank_db.transactions
    GROUP BY 
        YEAR
    ORDER BY 
        Year;
    \end{MySQLCode}

    \item Truy vấn số lượng tiền vào mỗi tháng (group by + aggregate functions)
    \begin{MySQLCode}
    SELECT 
        DATE_FORMAT(time, '%Y-%m') AS month, 
        SUM(amount) AS totalAmount
    FROM 
        bank_db.transactions
    WHERE 
        amount IS NOT NULL
        AND transactionType IN ('DEPOSIT')
    GROUP BY 
        month
    ORDER BY 
        totalAmount;
    \end{MySQLCode}
    
    \item Truy vấn số lượng tiền ra mỗi tháng (group by + aggregate functions)
    \begin{MySQLCode}
    SELECT 
        DATE_FORMAT(time, '%Y-%m') AS month, 
        SUM(amount) AS totalAmount
    FROM 
        bank_db.transactions
    WHERE 
        amount IS NOT NULL
        AND transactionType = 'WITHDRAW'
    GROUP BY 
        month
    ORDER BY 
        totalAmount;
    \end{MySQLCode}

    \item Truy vấn tổng số tiền của mỗi loại giao dịch (group by + aggregate functions)
    \begin{MySQLCode}
    SELECT 
        transactionType, 
        SUM(amount) AS totalAmount
    FROM 
        bank_db.transactions
    WHERE 
        amount IS NOT NULL
    GROUP BY 
        transactionType
    ORDER BY 
        totalAmount DESC;
    \end{MySQLCode}

\end{itemize}
TRUY VẤN CHI NHÁNH:
\begin{itemize}

     \item Truy vấn mỗi chi nhánh có tất cả bao nhiêu nhân viên (inner join + group by + aggregate functions)
    \begin{MySQLCode}
    SELECT 
        b1.branchCode,
        b1.branchName,
        COUNT(e1.employeeCode) AS totalEmployee
    FROM 
        bank_db.branches b1
    JOIN 
        bank_db.employees e1 ON b1.branchCode = e1.branchCode
    GROUP BY 
        b1.branchCode;
    \end{MySQLCode}

    \item Truy vấn tổng số tiền các giao dịch rút tiền, chuyển tiền, nạp tiền đã thực hiện tại mỗi chi nhánh (inner join + group by + aggregate functions)
    \begin{MySQLCode}
    SELECT 
        b.branchName AS branchName,
        t.transactionType,
        SUM(t.amount) AS totalAmount
    FROM 
        bank_db.transactions t
    JOIN 
        bank_db.employees e ON t.employeeCode = e.employeeCode
    JOIN 
        bank_db.branches b ON e.branchCode = b.branchCode
    WHERE 
        t.amount IS NOT NULL
        AND t.transactionType IN ('WITHDRAW', 'TRANSFER', 'DEPOSIT')
    GROUP BY 
        b.branchName, t.transactionType
    ORDER BY 
        b.branchName, t.transactionType;
    \end{MySQLCode}

    \item Truy vấn số tài khoản được mở tại mỗi chi nhánh (inner join + group by + aggregate functions)
    \begin{MySQLCode}
    SELECT 
	a.openingBranch, 
        b.branchName,
        COUNT(*) AS totalAccounts
    FROM 
	bank_db.accounts a
    JOIN
	bank_db.branches b ON a.openingBranch = b.branchCode
    GROUP BY
	openingBranch;
    \end{MySQLCode}

    \item Truy vấn số nhân viên trong từng chức vụ tại mỗi chi nhánh (inner join + group by + aggregate functions)
    \begin{MySQLCode}
    SELECT 
	b.branchCode,
        b.branchName,
        e.position,
        COUNT(*) AS totalEmployees
    FROM 
        bank_db.employees e
    JOIN 
        bank_db.branches b ON e.branchCode = b.branchCode
    GROUP BY 
        b.branchCode, e.position;
    \end{MySQLCode}

    \item Truy vấn các chi nhánh và tên đầy đủ của giám đốc chi nhánh đó (inner join + aggregate functions)
    \begin{MySQLCode}
    SELECT 
	b.branchCode,
        b.branchName,
        CONCAT(e.lastName, ' ', e.firstName) as directorName
    FROM 
        bank_db.employees e
    JOIN 
        bank_db.branches b ON e.branchCode = b.branchCode
    WHERE
    	e.position = 'Giám đốc chi nhánh'
    \end{MySQLCode}

    \item Truy vấn số lượng chi nhánh tại mỗi tỉnh thành (group by + aggregate functions)
    \begin{MySQLCode}
    SELECT 
	province,
        COUNT(*) AS totalBranches
    FROM 
        bank_db.branches
    GROUP BY 
        province;
    \end{MySQLCode}

\end{itemize}
TRUY VẤN SẢN PHẨM:
\begin{itemize}

    \item Truy vấn mỗi sản phẩm có bao nhiêu tài khoản (inner join + group by + aggregate functions)
    \begin{MySQLCode}
    SELECT
	p.productCode,
        p.productType,
        COUNT(*) AS totalAccounts
    FROM 
	bank_db.products p
    JOIN
	bank_db.accounts a ON a.productCode = p.productCode
    GROUP BY
	p.productCode;
    \end{MySQLCode}

    \item Truy vấn mỗi khách hàng yêu thích sản phẩm nào nhất (giả sử sản phẩm nào khách hàng có nhiều tài khoản nhất thì là yêu thích nhất)(inner join + subquery trong from + group by + aggregate functions)
    \begin{MySQLCode}
    SELECT
        c.customerID,
        CONCAT(c.lastName, ' ', c.firstName) AS customerName,
        p.productCode,
        p.productType,
        COUNT(*) AS totalAccounts
    FROM
        bank_db.customers c
    JOIN 
        bank_db.accounts a ON c.customerID = a.customerID
    JOIN 
        bank_db.products p ON a.productCode = p.productCode
    GROUP BY
        c.customerID,
        p.productCode,
        p.productType
    HAVING
        COUNT(*) = (
            SELECT 
                MAX(account_count)
            FROM (
                SELECT 
                    c2.customerID,
                    p2.productCode,
                    COUNT(*) AS account_count
                FROM
                    bank_db.customers c2
                JOIN 
                    bank_db.accounts a2 ON c2.customerID = a2.customerID
                JOIN 
                    bank_db.products p2 ON a2.productCode = p2.productCode
                GROUP BY
                    c2.customerID,
                    p2.productCode
            ) AS subquery_max
            WHERE subquery_max.customerID = c.customerID
        );
    \end{MySQLCode}

\end{itemize}

    
    

    

